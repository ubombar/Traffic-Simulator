\documentclass[a4paper, 12pt]{article}

\usepackage[margin=1.0in]{geometry}
\usepackage[backend=bibtex,style=verbose-trad2]{biblatex}
\usepackage[toc, numberedsection]{glossaries}


\title{Traffic Simulator: Analysis Report}
\date{August 12, 2019}
\author{Ufuk Bombar}

\makenoidxglossaries
\newglossaryentry{Python}{name=Python,description={A high level programming language that is mostly used on data analytics, machine learning and other high level fields}}
\newglossaryentry{}{name={},description={}}

\bibliography{sample}

\begin{document}
    \maketitle
    \newpage
    \tableofcontents
    \newpage

    \section{Introduction}

    \section{System Proposal}
        This section briefly covers what system should be able to do and how would it react on different scenerios.

        \subsection{Functional Requirements} \paragraph{}
            This section covers the most important necessities of software. These basic princibles are required 

            \subsubsection{Visualization of Data} \paragraph{}
                Vizualization is a porblem 

            \subsubsection{Custimization of Source} \paragraph{}
            \subsubsection{Real Time Display} \paragraph{}
        \subsection{Nonfunctional Requirements} \paragraph{}
            \subsubsection{Usability} \paragraph{}
            \subsubsection{Performance} \paragraph{}
                Performance may be a bottleneck on \gls{Python}. But most of the simulation constrains will be calculated on preprocessing, therefore the impact of \gls{Python} to performance wont be high.
            \subsubsection{Availability} \paragraph{}
                
            \subsubsection{Data Integrity} \paragraph{}
    \section{System Models} \paragraph{}

    \printnoidxglossaries

    \printbibliography[
        heading=bibnumbered,
        title={References}]
    
\end{document}